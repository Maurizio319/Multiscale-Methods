% !TeX root = ../main.tex
\subsection{Trajectories analysis} \label{chap: trajectory analysis}

The analysis of the trajectories was done by taking together and considering as independent the results obtained from each single chain of each replicate. For this reason, it was also decided to put together all the outputs and produce the collective plots represented in figures \ref{fig:RMSD collective replicates}, \ref{fig:RG collective replicates} and \ref{fig:autocorrelation function collective replicates}.

Three types of analysis were performed: 

\begin{enumerate}
    \item \textbf{\textit{Root Mean Square Deviation (RMSD)}}: The RMSD is evaluated by using the \textit{MDanalysis} package
    \cite{gowersMDAnalysisPythonPackage2016} (\textit{rmsd} in \textit{MDAnalysis.analysis.rms})
    and is calculated as follows:

    \begin{equation} \label{eq: RMSD}
        RMSD = \rho(t) = \sqrt{\frac{1}{N} \sum_{i=1}^N{w_i \left(\vec{x_i}(t) - \vec{x_i}^{REF}\right)^2}}
    \end{equation}

    Before performing this type of calculation, the structures were aligned to the first frame (each frame of each replicate was aligned to the first frame of the replicate). This type of alignment was done making use of the \textit{AlignTraj} function
    \cite{gowersMDAnalysisPythonPackage2016}
    . When interpreting the results obtained from RMSD calculation, it is generally considerable true the concept that the smaller is the difference between two structures, the lower is the value of RMSD. The results are written in graph \ref{fig:RMSD first replicate} and \ref{fig:RMSD collective replicates}.
    

    %%%%%%%%%%%%%%%%%%%%%%%%%%%%%%%%%%%%%%%%%%%%%%%%%%%%%%%%%%%%%%%%%%%%%%%%%%%%%%%%%%%%%%

    \item \textbf{$R_g$}: The Radius of Gyration was calculated as written in equation \ref{eq: radius of gyration} through the \textit{MDanalysis} package (\textit{radius\_of\_gyration} function). This quantity is a measure of how the mass of an object is spread out relative to a particular axis of rotation. In general, it tells "how spherical" is an object
    \cite{gowersMDAnalysisPythonPackage2016,tuckermanStatisticalMechanicsTheory2015}
    ; the higher is the value of the radius of gyration, the lower is the sphericity of the substance. The results are written in graph \ref{fig:RG first replicate} and \ref{fig:RG collective replicates}.
    

    \begin{equation} \label{eq: radius of gyration}
        R_g = \sqrt{\frac{\sum_i{m_i \vec{r_i}^2}}{\sum_i{m_i}}}    
    \end{equation}


    \item \textit{\textbf{Autocorrelation function:}} The autocorrelation function represents the degree of similarity between a given time series and a lagged version of itself over successive time intervals. It can be written as shown in equation \ref{eq: autocorrelation}
    \cite{sumaElectricFieldDrivenTrappingPolyelectrolytes2018}. The results are depicted in graphs \ref{fig:autocorrelation function first replicate} and \ref{fig:autocorrelation function collective replicates}. 
    
    \begin{equation} \label{eq: autocorrelation}
      r_k = \frac{C_k}{C_0} = \frac{\frac{1}{M} \sum_{t = 1}^{M - k}{(A_t - \bar{A})(A_{t+k} - \bar{A})}}{\frac{1}{M} \sum_{t = 1}^{M - k}{(A_t - \bar{A})^2}}
    \end{equation}

    Where $C_k = \frac{1}{M} \sum_{t = 1}^{M - k}{(A_t - \bar{A})(A_{t+k} - \bar{A})}$ is the autocovariance function at lag k and $C_0 = \frac{1}{M} \sum_{t = 1}^{M - k}{(A_t - \bar{A})^2}$ is the variance function.
    
\end{enumerate}

%#TODO PCA analysis
% \subsubsection{PCA analysis}
% A PCA analysis was performed for the trajectories with the aim of finding
