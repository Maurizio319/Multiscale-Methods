\subsection{ATAC-sequencing and CTCF sequencing}
ATAC-sequencing is a technology that allows for the identification of open-
chromatin regions 
\cite{buenrostroTranspositionNativeChromatin2013a, grandiChromatinAccessibilityProfiling2022}. In order to work, it requires the addition of TN-5,
a hyper-active transposase. The latter is preloaded with sequencing adapters
\cite{grandiChromatinAccessibilityProfiling2022}
to induce a contempourary reaction of fragmentation and ligation of the pieces released, in
a process called segmentation. The obtained adapted fragments are then amplified and sequenced. Once the reads are generated, a peak-calling algorithm (in our case MACS-2 
\cite{zhangModelbasedAnalysisChIPSeq2008a}
) is used to determine which portions of the genome present ATAC peaks, and areas where there are
significant enrichments of aligned reads with respect to the background. A significant enrich-
ment of reads is possible only in accessible regions, which are generally also the most active
ones and with available sites for transcription factors binding.
CTCF data, named in chapter ..., were obtained through a classical CHIP-sequencing.