% !TeX root = ../main.tex
\subsection{Chromatin as the information center of a cell}
All the living organisms have, inside the nucleus, the largest portion of their DNA, which is the main molecule through which information is passed from the old generation to the daughter cells. Due to the extreme length of the chromosomes, a coordinated assembly of DNA, proteins and RNA, called chromatin, is generated in an ordered and functional manner
\cite{paroBiologyChromatin2021}. 
The most important proteins used to reach this scope are histones, towards which DNA is wrapped around, forming the nucleosomes. To govern the functioning of the DNA, the histones and the DNA itself are subjected to a variety of modifications. Among those, methylation is the most important on  involving the nucleic acid. This type of modification, in mammals, occurs in specific sites of the genome, called CpGs, where a cytosine is connected directly to a guanine. Methylations of regulatory elements have been implicated in determining cell identity and chromatin structure
\cite{liauAdaptiveChromatinRemodeling2017a, shareefExtendedrepresentationBisulfiteSequencing2021}. 
On the other hand, CTCF is a protein conserved in eukariotes and is ubiquitous in mammalians
\cite{kimCTCFMultifunctionalProtein2015a}. 
It contains a Zync-finger which binds to DNA. The act of binding is performed in cooperation with cohesins, and causes the folding of the chromatin
\cite{hsiehEnhancerPromoterInteractions2022,kimCTCFMultifunctionalProtein2015a}.

