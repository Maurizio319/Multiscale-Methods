\graphicspath{{images/}}
\newpage
\section{ABSTRACT}

The chromatin is one of the most important parts of a cell. In fact, it contains in its volume the largest part of the cell DNA, and a great number of proteins, such as the histones, which adiuvate in the functional compaction of the nuclear DNA. However, the direct study of this substance encounters significant difficulties, and the analysis of related data do not give straightforward results. All-atomistic simulation approaches to predict the conformations of the chromatin in time are completely unfeasable, due to the large amount of atoms to simulate. Because of that, it is necessary to adopt a justified coarse-grained approach, which allows for simpler and less complex simulations. To this scope, I had the great pleasure of working in collaboration and on the code written by professor Marco Di Stefano. 
The aim of the project (which is still on-going) is to predict Hi-C matrices of contact by using the already named molecular dynamics simulations. Those maps are generally particularly hard to obtain and of high cost, however, the information that they contain can unveil very interesting mechanisms, such as the promoter-enhancer interactions.
At first, a decompaction and a relaxation trajectory were made for 100 replicates. Then, the matrices were produced by tuning the parameters used as weights for the potentials with an iterative-stepwise approach. The results were confronted in the process and at the end with experimentally obtained maps by computing the SCC metric.
To expand this study, we are already thinking about using the method for other \textit{loci} and cell-types.
Lastly, this report and its presentation could help importantly in improving the procedure and the analysis produced.
