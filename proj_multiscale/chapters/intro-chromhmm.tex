\subsection{\textit{ChromHMM} allows the sequential characterization of DNA regions} \label{intro: chromhmm}
\textit{ChromHMM} is a tool which helps in the annotation of genomic DNA by using epigenomic information
\cite{ernstChromatinstateDiscoveryGenome2017}.
It learns chromatin states signatures by using a multivariate hidden Markov model: in each genomic position (segment), it returns the most probable chromatin state and other useful information, such as the emission/transition parameters of the states, the abundance of the states at the TSS (Transcriptional Starting Site), at the TES (Transcriptional Ending site), and other relevant portions of the genome (CPG islands, exons, genes)
\cite{chilledhousevibesLearningChromatinStates2015,ernstChromatinstateDiscoveryGenome2017}. 

The package works through two functions in particular, which are the following
\cite{ernstChromatinstateDiscoveryGenome2017}:
\begin{enumerate}
    \item \textbf{\textit{BinarizeBam}}: it converts a set of \textit{.bam} files of aligned reads into binarized data files in a specified output directory, which can then be used as input to the \textit{LearnModel} function. When using this command, it has to be specified the bin size, that in the case of this project was set to be 200 bps.
    \item  \textbf{\textit{LearnModel}}: it takes a set of binarized data files, learns chromatin state models, and by default produces all the data already mentioned. Additionally, a webpage is created with links to all the files and images created.
\end{enumerate}

The results obtained are shown in chapter \ref{chap: ChromHMM results}.
