\chapter{Questions}

All in notes if not specified\\
\small{
Legend:
\begin{itemize}
    \item tuckerman: Tuckerman book
    \item not notes
    \item demonstrate: make the demonstration
\end{itemize}
}
\hfill \\
\small{
Lessons questions
\begin{itemize}
    \item Lesson 1
    \begin{enumerate}
        \item What is soft matter?
        \item Is active moving matter soft material?
        \item What is the definition of soft matter?: Soft matter or soft condensed matter is a subfield of condensed matter comprising a variety of physical systems that are deformed or structurally altered by thermal or mechanical stress of the magnitude of thermal fluctuations.
        \item What does it mean to define soft matter as mesoscopic? what are the consequences?
        \item How should be defined the bigger compounds? what happens to time and space scale when those are utilized?
        \item Will have the force to be specific?
        \item What is the approximation that  allow to simplify quantum mechanical problems with classical mechanics?
        \item How are thermal systems driven?
        \item What is the Helmothz free energy? Which system is described? How is it obtained? (Tuckerman)
        \item What is the critical point? (See tuckerman) What happens?
        \item What is the Ising model? How can you describe with it the critical point?
        \item What is necessary to make Multiscale methods relevant? (hierarchy)
        \item What procedure allows the coarse-graining?
    \end{enumerate}
    \item Lesson 2
    \begin{enumerate}
    \item Where are defined experimental observations?
        \item What is the Newton principle of determinacy
        \item What elements are defined in your experimental space?
        \item What is Galilean invariance? In which conditions it is maintained?
        \item What is a model? and what allows to build?
        \item Is invariance maintained in case of isolated systems? and in the case of systems with several bodies?
        \item When are forces conservative?
        \item What is the action integral? What does it tell? What does satisfy a path called stationary? (demonstrate) (tuckerman)
        \item What does the action integral concept suggest? (tuckerman)
        \item What is the formula of the Lagrangian?
        \item What is the Legendre transformation (tuckerman)
        \item What is the Hamiltonian? how is it obtained? (tuckerman)
        \item Which are the Hamilton equations?
        \item What happens to the Hamiltonian when the Lagrangian is conserved?
        \item What is an ensamble?
        \item How is the value of a macroscopic observable normally obtained? (through a time average ...)
        \item To what a time average can be converted? what is a phase space average?
        \item When is a stochastic process is called ergodic?
        \item How do we obtain a microcanonical ensamble?
        \item What are Poisson brackets? WHen is a quantity conserved? (tuckerman)
        \item What does the Liouville theorem say?
        \item When is the average of a quantity constant in time?
        \item How can you write the average of that quantity?
        \item What is a partition function? what type of coefficient it has?
        \item Formula for entropy in the microcanonical ensamble
    \end{enumerate}
    \item Lesson 3
    \begin{itemize}
        \item Describe the canonical partition function, what is the related coefficient?
        \item What is a Boltzmann weight?
        \item Write the Hamiltonian formualtion for a quantum problem
        \item What type of ansatz you assume to simplify your work? What is the diabatic approximatioin
        \item Describe the Born-Oppenheimer approximation
        \item What is meant by separability when talking about the potential energy? and additivity?
        \item What is a multi-body potential? What type of truncation you can perform?
        \item What is the difference between bonded and non-bonded interactions?
    \end{itemize}    
\end{itemize}
}